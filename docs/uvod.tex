\section{Uvod}

Razvojem novih tehnologija i uređaja, postoji kontinuirana potreba za napretkom računalne grafike, kako u mogućnostima tako i performansama. Iako je primarni fokus ovoga napretka bio foto realističan prikaz, ponekada je potreban drugačiji pristup iz praktičnih ili estetskih razloga. Kako bi grafička sučelja omogućila slobodu obrade slike, a u isto vrijeme i kontrolu nad performansama, koriste se programska sučelja - API \footnote{engl. Application programming interface - set definicija, protokola i alata za izradu programa}. Jedno od najpoznatijih grafičkih aplikacijski sučelja je OpenGL \footnote{engl. Open Graphics Library}.

Za prikaz 3D modela u realnom vremenu, potrebno je izvršiti niz zahtjevnih matematičkih operacija kako bi se odredio položaj i boja svake točke \footnote{engl. pixel - najmanja jedinica u računalnoj slici} na ekranu. Takve operacije najčešće nisu dovoljno brze na procesoru samog računala, pa se za te potrebe koristi specijalizirano sklopovlje grafičkih kartica. Pristup tom sklopovlju omogućen nam je preko OpenGL-a koji omogućava izvršavanje vlastitih programa izravno na grafičkim karticama, pri čemu se ostvarjue velika brzina rada zbog paralelnog načina obrade podataka.

U ovom radu opisano je kako se koristi moderni OpenGL za prikaz modela. Opisuje se način učitavanja podataka i njihov prijenos u memoriju grafičke kartice. Izradom vlastitih programa za sjenčanje \footnote{engl. shader}, prikazano je kako se vrši programiranje grafičkih kartica za obradu slike s ciljem postizanja sepcijalnih efekata.

\subsection{Zadatak diplomskog rada}

Cilj je razviti prilagođeni sustav sjenčanja za OpenGL koji će 3D modele prikazivati nefotorealistično u svrhu stvaranja dojma da su ručno nacrtani. Potrebno je razviti aplikaciju koja omogućava učitavanje  3D modela iz datoteke te ga prikazuje korištenjem prilagođenog sustava sjenčanja. Funkcionalnost razvijene aplikacije treba testirati prikazivanjem nekoliko 3D modela.
