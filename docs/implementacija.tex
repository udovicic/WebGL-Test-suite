\section{Implementacija}

Prethodna poglavlja opisivala su metodu \emph{toon shading}-a, te način implementacija koji se svodi na izradu \emph{shader} programa. No sam \emph{shader} program, izvodi se na grafičkoj kartici. U ovom poglavlju opisuje se sučelje koje je izrađeno za potrebe rada, koje krajnjem korisniku omogućava izmjenu modela i \emph{shader} programa. Opisuje se način implemntacija, te interakcija sa grafičkom karticom korištenjem \emph{three.js} biblioteke za rad s WebGL-om.

Aplikacija je izrađena pomoću HTML5 tehnologija i \emph{JavaScrip} programskog jezika. Na taj način, omogućeno je izvršavanje neovisno o platofrmi na većini modernih internet preglednika, te uređajima koji podržavaju HTML5 \emph{canvas} element. Budući da mogućnost izvršavanja aplikacija ovisi o grafičkom procesoru i karakteristikama samoga uređaja, korištena je biblioteka \emph{modernizer}\footnote{https://modernizr.com/} za određivanje sklopovske podrške samoga uređaja na kojemu se aplikacija izvršava.

\subsection{Korištenje \emph{three.js}-a}

Opisati framework i u čemu nam pomaže

\subsection{Učitavanje modela}

Na koji način se učitavaju modeli

\subsection{Korištenje više slojeva}

Multi-pass rendering u THREE.js-u