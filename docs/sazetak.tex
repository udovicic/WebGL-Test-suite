\section*{Sažetak}

U ovom diplomskom radu je opisan i programski ostvaren algoritam za \emph{toon shading} način sjenčanja 3D modela, kao i popratna korisnička aplikacija koja krajnjem korisniku omogućava izvršavanje \emph{shader} korisničkoga programa na grafičkoj kartici računala. Prikazan je način rada algoritma, kao i način implementacije korištenjem \emph{JavaScript} programskog jezika i \emph{JavaScript} biblioteke. Neke od prednosti ovakvoga pristupa su mogućnost izvršavanja programskoga koda neovisno o platformi i operativnom sustavu, te jednostavnost implementacije. Uočeni nedostaci algoritma su nemogućnost implmentacije u jednoj iteraciji, što u konačni negativno utječe na performanse aplikacije.

\textbf{Ključne riječi} sjenčanje 3D modela, detekcija rubova, OpenGL, WebGl, Toon shader, sjenčanje u više iteracija

\section*{Abstract}

This thesis describes and implements algorithm for \emph{toon shading} of 3D models, as well as a supporting application which enables end user to execute shader program on computers graphic card.  It also demonstrates how underlying algorithm works, as well as implementation of it using \emph{JavaScript} programming language and \emph{JavaScript} library. Some of the advantages of this approach are cross platform and operating system independent execution of application, as well as ease of implementation. Observed disadvantage of this approach is forced use of multipass rendering, which has negative impact on applications performance.


\textbf{Keywords} shading of 3D models, edge detection, OpenGL, WebGL, Toon shading, multipass rendering